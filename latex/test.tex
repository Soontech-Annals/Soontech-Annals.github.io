% Options here are passed to the article class.
% Most common options: 10pt, 11pt, 12pt
\documentclass[10pt]{datasheet2}

% Input encoding and typographical rules for English language
\usepackage[utf8]{inputenc}
\usepackage[english]{babel}
\usepackage[english]{isodate}

% tikz is used to draw images in this example, but you can
% also use \includegraphics{}.
\usepackage{graphicx}
\usepackage{float}
\usepackage{subcaption}

% These define global texts that are used in headers and titles.
\title{Welcome to Soontech!}
\author{Soontech Staff}
\tags{encoded-tech, public-archive, community-servers}
\date{25 December 2024}
\revision{Revision 1}
\begin{document}
\maketitle



\begin{multicols}{2}
\section{Features}
\begin{itemize}
\item{
   \href{https://github.com/Soontech-Annals/Archive}{\textbf{Public Archives}}: Over 50+ schematics and counting, with full LaTeX documentation for each device. This archive is a resource for anyone looking to learn about encoded storage tech in Minecraft.
}

\item{
   \href{https://discord.gg/rJ4W8RHrhe}{\textbf{Soontech Annals}}: A private Discord server which meets annually to curate content for the public archives. The most knowledgeable players in the community are members here.
}
\item{
   \href{https://discord.gg/dkSM2PyzJe}{\textbf{Soontech Now!}}: A community-run public Discord server for developing and learning about encoded storage technology in Minecraft. This server is open to anyone, and features weekly classes and public Minecraft servers for practical learning.
}
\end{itemize}

\columnbreak

\begin{figure}[H]
    \centering
    \includegraphics[width=0.4\textwidth]{soontm2.png}
    \caption{\centering Soontech Logo}
\end{figure}

\end{multicols}

\section{General Description}
Have you ever wanted to learn more about engineering advanced logistics technology in Minecraft? Well, you have come to the right place!

We at Soontech are dedicated to sharing our knowledge, so that you too can design and build your own encoded storage systems regardless of your current experience level. We have a wide range of resources available, from documented schematics to weekly in-game classes. Our community of experts come from a variety of backgrounds in STEM and have years of experience in Minecraft. Chill out with us in our Discord servers, or check out our public archives to get started!

\subsection{What is Encoded Storage?}

An encoded storage is a system that uses redstone signals to sort and store items in Minecraft. By melding principles from computer science, industrial engineering, and logistics in Minecraft, encoded storage systems allow you to automate your item management like never before. 

\begin{itemize}
    \item Extremely fast sorting using dynamic sorting methods and parallel processing
    \item Remote call systems that allow you to access your items from anywhere in the world through point to point networks
    \item Autocrafting systems that manufacture your supplies automatically
    \item And much more!
\end{itemize}




\end{document}

